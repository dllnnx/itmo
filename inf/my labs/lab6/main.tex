\thispagestyle{empty} % нет номеров страниц
\documentclass[12pt,a4paper]{report}
\usepackage[T1,T2A]{fontenc}
\usepackage[utf8x]{inputenc}
\usepackage[10pt]{extsizes}
\usepackage{geometry}
\usepackage{xcolor}
\usepackage{tikz}
\usepackage{graphicx, caption}
\usepackage{wrapfig}
\usepackage{amssymb}


\geometry{top=4em,right=2em,left=2em,bottom=4em}

\begin{document}
\newcommand{\Mainclt}{\raisebox{0pt}[\headheight][30pt]{\vbox{\hbox to\textwidth{74\hfilКНИГА I ПРЕДЛ. XLVIII. ТЕОРЕМА\hfil}}}}

\begin{minipage}[t]{0.3\textwidth}
  \hfill \\
  \begin{center}
    \begin{tikzpicture}
    \draw[orange,fill=orange] (0,-4) --  (.7,-4) arc(0:90:0.7) -- cycle;
    \draw[yellow, fill=yellow] (0,-4) -- (-.7, -4) arc(180:90:0.7) -- cycle;
      \draw[blue, ultra thick] (0,0) -- (0,-4);
      \draw[orange, ultra thick] (0,0) -- (2.5, -4);
      \draw[black, ultra thick] (2.5, -4) -- (0, -4);
      \draw[dashed, orange, ultra thick] (0,0) -- (-2.5, -4);
      \draw[dashed, black, ultra thick] (0,-4) -- (-2.5,-4);
      \node[above] at (0,0) {{\tiny$B$}};
      \node[below] at (0,-4) {{\tiny$A$}};
      \node[right] at (2.5, -4) {{\tiny$C$}};
      \node[left] at (-2.5, -4) {{\tiny$D$}};
    
    \end{tikzpicture}
    
  \end{center}
\end{minipage}
\hfil
\begin{minipage}[t]{0.57\textwidth}
  \Mainclt
  \begin{wrapfigure}{l}{0.2\linewidth}
    \includegraphics[width=0.9\linewidth]{E.jpg}
  \end{wrapfigure}\\
  \slshape сли \textit{в треугольнике квадрат одной стороны
  \begin{tikzpicture}
    \draw[orange,ultra thick] (0,0) -- (1,0);
    \node[above] at (0,0) {{\tiny$B$}};
    \node[above] at (1,0) {{\tiny$C$}};
  \end{tikzpicture} равен сумме квадратов двух дру- \\
  гих сторон
  \begin{tikzpicture}
    \draw[blue,ultra thick] (0,0) -- (1,0);
    \node[above] at (0,0) {{\tiny$A$}};
    \node[above] at (1,0) {{\tiny$B$}};
  \end{tikzpicture} и
  \begin{tikzpicture}
    \draw[black,ultra thick] (0,0) -- (1,0);
    \node[above] at (0,0) {{\tiny$A$}};
    \node[above] at (1,0) {{\tiny$C$}};
  \end{tikzpicture}, то угол
    \begin{tikzpicture}
        \draw[orange,fill=orange] (-0.7,0) --  (0,0) arc(0:90:0.7) -- cycle;
        \node[left] at (-0.7, 0.7) {{\tiny$B$}};
        \node[left] at (-0.7, 0){{\tiny$A$}};
        \node[right] at (0, 0) {{\tiny$C$}};
    \end{tikzpicture}, заключенный между этими двумя сторонами прямой.}
  
  \upshape
  \begin{center}
   Проведем
    \begin{tikzpicture}
      \draw[dashed, black,ultra thick] (0,0) -- (1,0);
    \node[above] at (0,0) {{\tiny$A$}};
    \node[above] at (1,0) {{\tiny$D$}};
  \end{tikzpicture}
  $\perp$
    \begin{tikzpicture}
        \draw[blue,ultra thick] (0,0) -- (1,0);
        \node[above] at (0,0) {{\tiny$A$}};
        \node[above] at (1,0) {{\tiny$B$}};
    \end{tikzpicture} \\
     и =
    \begin{tikzpicture}
    \draw[black,ultra thick] (0,0) -- (1,0);
    \node[above] at (0,0) {{\tiny$A$}};
    \node[above] at (1,0) {{\tiny$B$}};
  \end{tikzpicture} (пр. I.{\tiny II}, I.{\tiny3}),\\
    также проведем
    \begin{tikzpicture}
      \draw[dashed, orange, ultra thick] (0,0) -- (1,0);
    \node[above] at (0,0) {{\tiny$B$}};
    \node[above] at (1,0) {{\tiny$D$}};
  \end{tikzpicture}.\\
    \vspace{0.4cm}
    Поскольку
    \begin{tikzpicture}
      \draw[dashed, black,ultra thick] (0,0) -- (1,0);
    \node[above] at (0,0) {{\tiny$A$}};
    \node[above] at (1,0) {{\tiny$D$}};
  \end{tikzpicture} =
   \begin{tikzpicture}
    \draw[black,ultra thick] (0,0) -- (1,0);
    \node[above] at (0,0) {{\tiny$A$}};
    \node[above] at (1,0) {{\tiny$C$}};
  \end{tikzpicture} (постр.)\\
  \[
    \begin{tikzpicture}
      \draw[dashed, black,ultra thick] (0,0) -- (1,0);
    \node[above] at (0,0) {{\tiny$A$}};
    \node[above] at (1,0) {{\tiny$D$}};
  \end{tikzpicture} ^{\text{2}} =
   \begin{tikzpicture}
    \draw[black,ultra thick] (0,0) -- (1,0);
    \node[above] at (0,0) {{\tiny$A$}};
    \node[above] at (1,0) {{\tiny$C$}};
  \end{tikzpicture} ^{\text{2}};\]
  \vspace{0.4cm}
  \begin{math}
  \therefore
  \begin{tikzpicture}
      \draw[dashed, black,ultra thick] (0,0) -- (1,0);
    \node[above] at (0,0) {{\tiny$A$}};
    \node[above] at (1,0) {{\tiny$D$}};
  \end{tikzpicture} ^{\text{2}} + 
  \begin{tikzpicture}
        \draw[blue,ultra thick] (0,0) -- (1,0);
        \node[above] at (0,0) {{\tiny$A$}};
        \node[above] at (1,0) {{\tiny$B$}};
    \end{tikzpicture} ^{\text{2}} =
    \begin{tikzpicture}
    \draw[black,ultra thick] (0,0) -- (1,0);
    \node[above] at (0,0) {{\tiny$A$}};
    \node[above] at (1,0) {{\tiny$C$}};
  \end{tikzpicture} ^{\text{2}} +
  \begin{tikzpicture}
        \draw[blue,ultra thick] (0,0) -- (1,0);
        \node[above] at (0,0) {{\tiny$A$}};
        \node[above] at (1,0) {{\tiny$B$}};
    \end{tikzpicture} ^{\text{2}}
  \end{math} \\
  но
   \begin{math}
    \begin{tikzpicture}
      \draw[dashed, black,ultra thick] (0,0) -- (1,0);
    \node[above] at (0,0) {{\tiny$A$}};
    \node[above] at (1,0) {{\tiny$D$}};
  \end{tikzpicture} ^{\text{2}} + 
  \begin{tikzpicture}
        \draw[blue,ultra thick] (0,0) -- (1,0);
        \node[above] at (0,0) {{\tiny$A$}};
        \node[above] at (1,0) {{\tiny$B$}};
    \end{tikzpicture} ^{\text{2}} = 
    \begin{tikzpicture}
      \draw[dashed, orange, ultra thick] (0,0) -- (1,0);
    \node[above] at (0,0) {{\tiny$B$}};
    \node[above] at (1,0) {{\tiny$D$}};
  \end{tikzpicture} ^{\text{2}}
  \end{math} (пр.I.{\tiny47}), \\
  и
  \begin{math}
      \begin{tikzpicture}
    \draw[black,ultra thick] (0,0) -- (1,0);
    \node[above] at (0,0) {{\tiny$A$}};
    \node[above] at (1,0) {{\tiny$C$}};
  \end{tikzpicture} ^{\text{2}} +
  \begin{tikzpicture}
        \draw[blue,ultra thick] (0,0) -- (1,0);
        \node[above] at (0,0) {{\tiny$A$}};
        \node[above] at (1,0) {{\tiny$B$}};
    \end{tikzpicture} ^{\text{2}} =
     \begin{tikzpicture}
    \draw[orange,ultra thick] (0,0) -- (1,0);
    \node[above] at (0,0) {{\tiny$B$}};
    \node[above] at (1,0) {{\tiny$C$}};
  \end{tikzpicture}
  \end{math} (гип.)\\
  \vspace{0.4cm}
  \begin{math}
      \therefore
      \begin{tikzpicture}
      \draw[dashed, orange, ultra thick] (0,0) -- (1,0);
    \node[above] at (0,0) {{\tiny$B$}};
    \node[above] at (1,0) {{\tiny$D$}};
  \end{tikzpicture} ^{\text{2}} = 
  \begin{tikzpicture}
    \draw[orange,ultra thick] (0,0) -- (1,0);
    \node[above] at (0,0) {{\tiny$B$}};
    \node[above] at (1,0) {{\tiny$C$}};
  \end{tikzpicture} ^{\text{2}}
  \end{math}, \\
  \vspace{0.4cm}
  $\therefore$
  \begin{tikzpicture}
      \draw[dashed, orange, ultra thick] (0,0) -- (1,0);
    \node[above] at (0,0) {{\tiny$B$}};
    \node[above] at (1,0) {{\tiny$D$}};
  \end{tikzpicture} = 
  \begin{tikzpicture}
    \draw[orange,ultra thick] (0,0) -- (1,0);
    \node[above] at (0,0) {{\tiny$B$}};
    \node[above] at (1,0) {{\tiny$C$}};
  \end{tikzpicture}; \\
  \vspace{0.4cm}
  и $\therefore$    
  \begin{tikzpicture}
        \draw[yellow, fill=yellow] (0,-1) -- (-.7, -1) arc(180:90:0.7) -- cycle;;
        \node[left] at (-0.7, -1) {{\tiny$D$}};
        \node[right] at (0, -1){{\tiny$A$}};
        \node[above] at (0, -0.3) {{\tiny$B$}};
    \end{tikzpicture} = 
    \begin{tikzpicture}
        \draw[orange,fill=orange] (-0.7,0) --  (0,0) arc(0:90:0.7) -- cycle;
        \node[above] at (-0.7, 0.7) {{\tiny$B$}};
        \node[left] at (-0.7, 0){{\tiny$A$}};
        \node[right] at (0, 0) {{\tiny$C$}};
    \end{tikzpicture}
    (пр. I.8), \\
    \vspace{0.4cm}
    следовательно
    \begin{tikzpicture}
        \draw[orange,fill=orange] (-0.7,0) --  (0,0) arc(0:90:0.7) -- cycle;
        \node[above] at (-0.7, 0.7) {{\tiny$B$}};
        \node[left] at (-0.7, 0){{\tiny$A$}};
        \node[right] at (0, 0) {{\tiny$C$}};
    \end{tikzpicture}
    прямой угол. 
  
  \end{center}
  \begin{flushright}
    ч. т. д.
  \end{flushright}
\end{minipage}
\end{document}